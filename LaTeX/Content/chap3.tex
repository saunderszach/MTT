In this section, we deal with the case of no detection ambiguity. Therefore, we add the following, more restrictive assumptions, to those presented in Assumption~\ref{ass:general_assumption}.

%%% IEEE %%%
\begin{assumption}\label{ass:basic_assumptions}
\leavevmode
\begin{enumerate}[(i)]
\item The sensor generates exactly one detection for each target
 at each scan i.e., no missed detections.
\item The sensor does not generate any additional detections
i.e., no false alarms.
\end{enumerate}
\end{assumption}

%%% Thesis %%%
%\begin{assumption}\label{ass:basic_assumptions}
%\begin{minipage}[t]{\columnwidth}
%\begin{enumerate}[(i)]
%\item The sensor generates exactly one detection for each target\\
% at each scan i.e., no missed detections.
%\item The sensor does not generate any additional detections\\
%i.e., no false alarms.
%\end{enumerate}
%\end{minipage}
%\end{assumption}

We begin constructing our MIO model by introducing decision variables that define data associations as well as estimated trajectories. Using these decision variables, we then develop an objective function that  mathematically quantifies the value of the model decisions. Finally, we restrict these variables using constraints that force the model to find solutions that are feasible for the MTT problem as we have defined it. A simple model is first developed step by step in the coming sections before generalized formulation is presented. 

\subsection{Decision Variables}
The data association and trajectory estimation problems require unique decision variables. Because these two problems lie in different domains, the variables we use to represent these decisions also differ. First, we introduce \textit{continuous} decision variables $\alpha_{j} \in \mathbb{R}^n$ and $\beta_{j} \in \mathbb{R}^n$ to represent the estimated initial position and velocity, respectively, of each trajectory \textit{j}. In our interpretation of the MTT problem we allow the trajectory parameters to lie anywhere in the real-continuous domain. For the data estimation problem, we wish to assign detections to trajectories, a naturally discrete problem. Therefore, we introduce binary decision variables $y_{itj}$ to indicate whether detection $x_{it}$ is assigned to trajectory \textit{j} or not:

\begin{align}
y_{itj} =
\begin{cases}
1, & \text{if detection $x_{it}$ is assigned to trajectory \textit{j},} \\
0, & \text{otherwise.}
\end{cases}
\end{align}

Next, we use these decision variables to develop an objective function to score the solutions found by the model. 

\subsection{Objective Function}
Here, we would like to develop a function that quantifies the quality of a feasible solution. Our goal is to produce a single measure for both the data association and the trajectory estimation problems. For a given assignment and a given estimated trajectories we define the quality of the estimation as the distance of each detection from the estimated position of its associated trajectory. That is, if at scan $t$ detection $x_{it}$ is associated with trajectory $j$ then, the distance 
$$\|x_{it}-\hat{x}_{jt}\|,$$
is the measure of the quality of estimation for trajectory $j$ at scan $t$, and the total estimation quality for a given association will be given as 
\begin{align}\label{eq:objective_base}
\sum_{j=1}^P\sum_{t=1}^T\left\|\sum_{(i,j)\in \mathcal{A}_{t}} x_{it} - \hat{x}_{jt}\right\|,
\end{align} 
where $\mathcal{A}_t$ is pairs of detection-trajectory associations made for scan $t$. 

We can now separate the problem into two parts: given an assignment finding the estimated trajectories which minimizes \eqref{eq:objective_base}, and finding the assignment which results in the best estimated trajectories.

Recall that each trajectory is defined by two parameters, $\alpha^{\text{true}}_{j}$ and $\beta^{\text{true}}_{j}t$ such that the true location is given as 
\begin{align}
	\bar{x}_{jt} = \alpha^{\text{true}}_{j} + \beta^{\text{true}}_{j}t.
\end{align}
Thus, an estimated trajectory can analogically be defined by  $\alpha_{j}$ and $\beta_{j}$ such that its estimated location at the time of scan $t$ is given by
\begin{align}
	\hat{x}_{jt} =  \alpha_{j} + \beta_{j}t.
\end{align}
Therefore, given association $\mathcal{A}=(\mathcal{A}_1,\ldots,\mathcal{A}_T)$, the trajectory which has the best estimation error is actually the one which solves the problem
\begin{align}\label{eq:objective_mintraj}
\underset{\alpha_{j}, \beta_{j}}{\text{minimize: }}\sum_{j=1}^P\sum_{t=1}^T\left\|\sum_{(i,j)\in \mathcal{A}_{t}} x_{it} - (\alpha_{j} + \beta_{j}t)\right\|,
\end{align} 
Notice that under the current assumptions, in which there is no detection ambiguity, \eqref{eq:objective_mintraj} is the cost of association $\mathcal{A}$. 

Now we turn to the problem of choosing the assignment, based on this measure. To this end we formulate the assignment cost \eqref{eq:objective_mintraj} in terms of our decision variables. Note that $(i,j)\in\mathcal{A}_t$ if and only if $y_{itj}=1$, and because all detections will be assigned to a trajectory and vice versa, the following equivalence holds:

\begin{align}
\sum_{(i,j)\in \mathcal{A}_{t}} x_{it} = \sum_{i=1}^{P}y_{itj}x_{it}.
\end{align}

Making the appropriate substitutions, the cost of an assignment described by variables $y_{itj}$ is given as
\begin{align}\label{eq:assignemnt_cost}
 \underset{\alpha_{j}, \beta_{j}}{\text{minimize: }} \sum_{j=1}^{P} \sum_{t=1}^{T}  \left \| \sum_{i=1}^{P}y_{itj}x_{it} x_{it} - \hat{x}_{jt} \right \|,
\end{align}
Therefore, in order to find the assignment with the lowest cost, we are left to minimize cost \eqref{eq:assignemnt_cost} over all assignments, and obtain the following final objective 
\begin{align}\label{eq:full_objective}
 \underset{y_{itj}, \alpha_{j}, \beta_{j}}{\text{minimize: }} \sum_{j=1}^{P} \sum_{t=1}^{T}  \left \| \sum_{i=1}^{P}y_{itj}x_{it} x_{it} - \hat{x}_{jt} \right \|,
\end{align}


At this point it is necessary to discuss the advantages and disadvantages of the two natural distance measures (norms) that will be considered: the $\ell_1$ and the $\ell_2$ norms. The $\ell_1$ norm has the advantage that it can be reformulated using linear optimization (through the addition of continuous variables and constraints), and it is well known to be more robust to outliers. Furthermore, existing algorithms for MIO are more well developed for linear rather than quadratic optimization. However, the $\ell_2$ norm squared form, which is equivalent to the residual sum of squares (RSS), has the advantage that it can be quickly computed using simple linear algebra, making it more amenable to a heuristic. This concept will be discussed further in \mysection~\ref{\myabrv Heuristic}.

Because of the computational benefits of linear optimization over quadratic optimization, we choose to formulate the objective using the $\ell_1$ norm. Therefore, we now show how \eqref{eq:full_objective} can be reformulated using linear optimization in the case of the $\ell_1$ norm by introducing continuous variables $\psi_{jt} \in \mathbb{R}^n$ and the following constraints,
\begin{align}
\sum_{i=1}^{P}y_{itj}x_{it} - \alpha_{j} - \beta_{j}t \leq \psi_{jt}, \qquad \forall j,t,\\
-\left(\sum_{i=1}^{P}y_{itj}x_{it} - \alpha_{j} - \beta_{j}t\right), \geq \psi_{jt} \qquad \forall j,t.
\end{align}
The resulting objective function for the case of the $\ell_1$ norm would then be:
\begin{align}
\underset{\psi_{jt}}{\text{minimize: }} & \sum_{j=1}^{P} \sum_{t=1}^{T} \psi_{jt}.
\end{align}

%For the case of the L2 norm, the objective function would be:
%\begin{align}
%\underset{\psi_{jt}}{\text{minimize: }} & \sum_{j=1}^{P} \sum_{t=1}^{T} {\|\psi_{jt}\|}^{2}_{2}
%\end{align}

\subsection{Constraints}
For each scan, each detection $x_{it}$ must be assigned to exactly one target \textit{j}:
\begin{align}\label{eq:all_detections}
\sum_{j=1}^{P} y_{itj} = 1 \qquad \forall i,t
\end{align}

Similarly, for each scan, each target must be assigned exactly one detection:
\begin{align}\label{eq:all_targets}
\sum_{i=1}^{P} y_{itj} = 1 \qquad \forall j,t
\end{align}

\subsection{Simple Formulation}
Combining all of these elements together, we arrive at the following MIO model:
\begin{align*}
\underset{\psi_{jt}}{\text{minimize: }} & \sum_{j=1}^{P} \sum_{t=1}^{T} \psi_{jt} \\
\text{subject to: }	& \sum_{j=1}^{P} y_{itj} = 1 \qquad \forall i,t\\
				& \sum_{i=1}^{P} y_{itj} = 1 \qquad \forall j,t\\
				& \sum_{i=1}^{P}y_{itj}x_{it} - \alpha_{j} - \beta_{j}t \leq \psi_{jt} \qquad \forall j,t\\
				& -\left(\sum_{i=1}^{P}y_{itj}x_{it} - \alpha_{j} - \beta_{j}t\right) \geq \psi_{jt} \qquad \forall j,t\\
			 	& y_{itj} \in \{0,1\} \quad \forall i,t,j\\
				& \alpha_{j} \in \mathbb{R}^n \quad \forall j,\quad \beta_{j} \in \mathbb{R}^n \quad \forall j, \quad z_{jt} \in \mathbb{R}^n \quad \forall j,t
\end{align*}

This formulation is simple in the sense that it involves few variables and constraints, making it highly interpretable and easily implementable. However, it has the disadvantage of being ill suited for extensions to detection ambiguity because it heavily relies on the fact that exactly one of the detections at each scan is associated to a target. This forces the term $\sum_{i=1}^{P}y_{itj}x_{it}$ to always be equal to one of the detections. However, in the case of detection ambiguity, this no longer holds true, since there might be trajectories which are not associated with a detection in a given scan, resulting in an unintended cost to the assignment. Therefore, in the following section we present a more generalized formulation, which is amenable to scenarios with detection ambiguity.

\subsection{Generalized Formulation}
We desire a formulation that can more easily extend to detection ambiguity. To this end, we introduce a new variable $z_{jt}$ which takes on the value $x_{it}$ when $y_{ijt}=1$ and some arbitrary number when $y_{itj}=0$. Using this method we must adjust \label{eq:simple_objective} as follows:

\begin{align}\label{eq:generalized_objective}
\underset{z_{jt}, \alpha_{j}, \beta_{j}}{\text{minimize: }} & \sum_{j=1}^{P} \sum_{t=1}^{T} \|z_{jt} - \alpha_{j} - \beta_{j}t\|
\end{align}

This objective can then be linearized by again introducing $\theta{jt}$ and similar constraints as follows. 
\begin{align}\label{eq:generalized_linear_objective}
\underset{\psi_{jt}}{\text{minimize: }} & \sum_{j=1}^{P} \sum_{t=1}^{T} \psi_{jt}
\end{align}
\begin{align}
z_{jt} - \alpha_{j} - \beta_{j}t \leq \psi_{jt} \qquad \forall i,j,t\\
-(z_{jt} - \alpha_{j} - \beta_{j}t) \geq \psi_{jt} \qquad \forall i,j,t
\end{align}
 
Furthermore, we must ensure that the decision variable $z_{jt}$ will only take on the value of $x_{it}$ in the objective function if $x_{it}$ is assigned to target \textit{j} ($y_{itj} = 1$). We enforce this effect using the following constraint:

\begin{align}
M_{t}(1-y_{itj}) \geq |z_{jt} - x_{it}y_{itj}| \qquad \forall i,t,j\\
\end{align}

where $M_{t} = \underset{j}{\text{max}}|x_{it}|$ for each scan. Furthermore, we can write this equivalently as a linear optimization problem by using the following set of two linear constraints:

\begin{align}
x_{it}y_{itj} + M_{t}(1-y_{itj}) \geq z_{jt} \qquad \forall i,t,j\\
x_{it}y_{itj} - M_{t}(1-y_{itj}) \leq z_{jt} \qquad \forall i,t,j
\end{align}

Combining all of these elements together, we arrive at the following generalized MIO model:
\begin{align*}
\underset{\psi_{jt}}{\text{minimize: }} & \sum_{j=1}^{P} \sum_{t=1}^{T} \psi_{jt} \\
\text{subject to: }	& \sum_{j=1}^{P} y_{itj} = 1 \qquad \forall i,t\\
				& \sum_{i=1}^{P} y_{itj} = 1 \qquad \forall j,t\\
				& x_{it}y_{itj} + M_{t}(1-y_{itj}) \geq z_{jt} \qquad \forall i,t,j\\
				& x_{it}y_{itj} - M_{t}(1-y_{itj}) \leq z_{jt} \qquad \forall i,t,j\\
				& z_{jt} - \alpha_{j} - \beta_{j}t \leq \psi_{jt} \qquad \forall i,j,t\\
				& -(z_{jt} - \alpha_{j} - \beta_{j}t) \geq \psi_{jt} \qquad \forall i,j,t\\
			 	& y_{itj} \in \{0,1\} \quad \forall i,t,j\\
				& \alpha_{j} \in \mathbb{R}^n \quad \forall j,\quad \beta_{j} \in \mathbb{R}^n \quad \forall j, \quad z_{jt} \in \mathbb{R}^n \quad \forall j,t
\end{align*}
