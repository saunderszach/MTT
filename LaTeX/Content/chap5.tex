We transition to treat the case of detection ambiguity. Specifically, we now allow for false alarms, the instance in which a detection is triggered when no target exists, and missed detections, the instance in which a target exists but no detection is generated. Consequently, the number of detections at each scan varies, and the number of targets we wish to track is now ambiguous. To state this explicitly, we introduce additional notation for the case of detection ambiguity. Let $n_{t}$ be the number of detections at scan \textit{t}. We denote 
\begin{align*}
N_{0} = \underset{t}{\text{min }} n_{t}
\end{align*}
as the largest number of detections across all scans, and similarly, we denote
\begin{align*}
N_{1} = \underset{t}{\text{max }}  n_{t}
\end{align*}
as the smallest number of detections across all scans. The only assumption we make in the case of detection ambiguity is that the true number of targets falls somewhere in the range of $[N_{0},N_{1}]$. Particularly, we replace Assumption~\ref{ass:basic_assumptions} with the following less restrictive assumption.
\begin{assumption}\label{ass:robust_assumptions}
\leavevmode
\begin{enumerate}[(i)]
\item The sensor generates at most one detection for each target
 at each scan i.e., there can be missed detections.
\item The sensor can generate detections which do not originate from any target
i.e., there can be false alarms.
\item The number of true targets $P$ satisfies $N_0\leq P \leq N_1$.
\end{enumerate}
\end{assumption}

Several new challenges emerge in the case of detection ambiguity. First, we now need to estimate the number of targets. We denote the number of estimated targets as $P_{\text{est}}$ Second, now each detection must either be assigned to a target, as before, or classified as a false alarm, not both. Finally, for each target we want to identify the scans in which it was not detected. 

In this section, we develop an approach that reduces the difficulty of estimating the number of targets inherent to this new problem by two different approaches.  In the first approach, the number of targets can be determined via the optimization model itself, through the use of additional decision variables and constraints. 
Alternatively, the second approach  we formulate an MIO model that assumes a fixed number of targets $P$ and then use the power of parallelization to run that model with all possible values of $P$ which satisfy Assumption~\ref{ass:robust_assumptions}, and then choose the solution for $P$ which minimizes the total objective function value. These two approaches are equivalent, in the sense they have the same optimal solution set.  
 However, as it turns out, the first approach leads to a model which is not tractable for practical use, while the second results in smaller more tractable models, which as we stated can be solved in parallel.
Therefore, we will now discuss in depth the second approach, which is an extension of model \eqref{eq:generalized_problem} to account for missed detections and false alarms, and defer the interested reader to Appendix~\ref{sec:Penalty_Appendix} for a complete discussion of this approach.

\subsection{Decision Variables}
Because we will assume a fixed number of targets $P$ for a given run of the the MIO, we only require two additional decision variables, one for identifying false alarms and another for identifying missed detections. Toward this goal, we introduce new binary decision variables $F_{it}$ to indicate whether or not a detection $x_{it}$ is a false alarm. 
\[F_{it} = 
\begin{cases}
1, & \text{if detection \textit{i} at time \textit{t} is a False Alarm,}\\
0, & \text{otherwise.}
\end{cases}\]
Likewise, we introduce binary decision variables $M_{jt}$ to indicate whether or not trajectory \textit{j} has a missed detection at time \textit{t}.
\[M_{jt} =
\begin{cases}
1, & \text{if detection for trajectory \textit{j}}\\
   &\text{at time \textit{t} is a Missed Detection,}\\
0, & \text{otherwise.}
\end{cases}\]

\subsection{Objective Function}
We can easily extend \eqref{eq:generalized_objective} to account for detection ambiguity by introducing penalties $\theta$ and $\phi$ for each missed detection and false alarm, respectively. This implies a linear penalty function, meaning that every missed detection (false alarm) is contributes the same penalty to the objective function. Therefore, we simply need to penalize the total number of false alarms and missed detections in the objective function. If we denote the total number of false alarms $TF$ and total number of missed detections $TM$, the resulting objective function takes the form:
\begin{align*}
\underset{\psi_{jt}}{\text{minimize: }} & \sum_{j=1}^{P} \sum_{t=1}^{T} \psi_{jt} + \theta \cdot TF + \phi \cdot TM.
\end{align*}
As a general rule, we would expect to increase $\theta$ or $\phi$, as the number of expected false alarms or missed detections decreases, respectively. A more exhaustive discussion on the insight behind these penalties, in addition to recommendations for tuning them can be found in Appendix~\ref{\myabrv Penalty_Appendix}.

\subsection{Constraints}
Finally, we must restrict the set of feasible solutions to satisfy the assumptions made earlier. We accomplish this by simply modifying \eqref{eq:all_detections} and \eqref{eq:all_targets} to account for false alarms and missed detections, respectively. Specifically, all detections must either be assigned to a trajectory \textit{j} or to a false alarm:
\begin{align}\label{eqn: FA Simple}
\sum_{j=1}^{P} y_{itj} + F_{it} = 1 \qquad \forall i,t.
\end{align}
All trajectories \textit{j} must either be assigned a detection or a missed detection:
\begin{align}
\sum_{i=1}^{n_{t}} y_{itj} + M_{jt} = 1 \qquad \forall j,t
\end{align}
Lastly, the sum of all false alarms must equal $TF$, and the sum of all missed detections must equal $TM$. 
\begin{align*}
\sum_{i=1}^{n_{t}} \sum_{t=1}^{T} F_{it} = TF
\end{align*}
\begin{align}\label{eqn: MD Total}
\sum_{j=1}^{P} \sum_{t=1}^{T} M_{jt} = TM 
\end{align}

\subsection{MIO with Fixed Number of Targets}
Merging all of these elements together we arrive at our MIO model for a case of detection ambiguity and a fixed number of targets. We refer to this model as the \textit{robust} MIO. 
\begin{align}
g(P)=\underset{\psi_{jt}}{\text{minimize: }} & \sum_{j=1}^{P} \sum_{t=1}^{T} \psi_{jt} + \theta \cdot TF + \phi \cdot TM \label{eq:simple_robust}\\
\text{subject to: }	& \sum_{j=1}^{P} y_{itj} + F_{it} = 1 \qquad \forall i,t \nonumber \\
				& \sum_{i=1}^{n_{t}} y_{itj} + M_{jt} = 1 \qquad \forall j,t \nonumber\\
				& \sum_{i=1}^{n_{t}} \sum_{t=1}^{T} F_{it} = TF \nonumber \\
				& \sum_{j=1}^{P} \sum_{t=1}^{T} M_{jt} = TM \nonumber \\
				& x_{it}y_{itj} + M_{t}(1-y_{itj}) \geq z_{jt} \qquad \forall i,t,j \nonumber \\
				& x_{it}y_{itj} - M_{t}(1-y_{itj}) \leq z_{jt} \qquad \forall i,t,j \nonumber \\
				& z_{jt} - \alpha_{j} - \beta_{j}t \leq \psi_{jt} \qquad \forall j,t \nonumber \\
				& -(z_{jt} - \alpha_{j} - \beta_{j}t) \leq \psi_{jt} \qquad \forall j,t \nonumber \\
				& y_{itj} \in \{0,1\} \quad \forall i,t,j \nonumber \\
				& \alpha_{j} \in \mathbb{R}^n,\quad \beta_{j} \in \mathbb{R}^n \quad \forall j \nonumber\\
				& z_{jt} \in \mathbb{R}^n, \quad \forall j,t. \nonumber
\end{align}

$P_\text{est}$ would then be given by
\begin{align}P_\text{est}=\min_{N_o\leq P\leq N_1} g(P),\end{align}
and the trajectories and assignments will correspond with the optimal solution of the corresponding model.

\subsection{Heuristic with Fixed Number of Targets}
The heuristic for the scenario with ambiguity follows similarly from the heuristic developed under the scenario without ambiguity. Mainly, the heuristic initializes in the same manner and sweeps through all scans continuously making random swaps. However, the key difference in the newly adapted heuristic is that false alarms and missed detections must be included in the switches. Therefore, the framework of the new algorithm, also referred to as the \textit{robust} heuristic, is the same as the one presented in Figure~\ref{fig:Basic_Heuristic}, but the robust heuristic randomly chooses from the following options when making swaps: 
\begin{enumerate}
  \item Switch detection assignments between two existing targets.
  \item Switch the detection assignment of an existing target with a false alarm.
  \item Switch the detection assignment of an existing target with a missed detection for a different existing target.
  \item Move the detection assignment of an existing target to a false alarm and replace it with a missed detection.
  \item Move a false alarm into the location of a missed detection for an existing target.
\end{enumerate}

Similar to the basic heuristic, this robust extension will accept the switch/move if the objective score improves, and reject the switch/move otherwise, and terminates once it completes a full sweep without accepting a single swap. Due to the increase in potential combinations of solutions, we expect this variant of the heuristic to run slightly slower. Though this effect is mitigated by the fact that this variant is just as paralellizable as the basic heuristic. 

Next, we shift our discussion to focus on measuring the complexity and performance of our approaches. 