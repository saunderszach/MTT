In this paper, we presented a new approach to the multi-target tracking problem which jointly solves the problems of data association and trajectory estimation via global optimization methods using a single objective function. Toward this goal, we proposed the use of a randomized local search heuristic as a warm start for a mixed integer optimization model, and we did so for scenarios with and without detection ambiguity. The approach is general since it makes no assumptions on the data generation process, and easily implementable, since it based on a simple model with little to no tunable parameters. We showed that the local search heuristic finds good quality feasible solutions very quickly, and that the MIO offers improvement over the heuristic in seconds. We also show that while the introduction of detection ambiguity deteriorates the performance of the data association, it does not significantly effect the quality of trajectory estimation. Moreover, the quality of the solution obtained is robust to the missed detection rate, but less so to the false alarm rate.
%We accomplish this with without the need of a trajectory bank nor the prior computation of trajectory hypotheses.

Though the methods have limited scalability for the scenarios examined, they show potential for use in a sliding window scheme, using past decision to be fixed and used as added information for the current tracking window. This will allow us to track targets in real time systems for longer periods.  %Implementing the heuristic as sliding window algorithm would very likely mitigate this scaling effect in regards to $T$. Rather than solve all scans in a single batch at once, a sliding window algorithm solves a subset of scans, or a smaller window, and advances through all scans sequentially.  As the window progresses forward through the scans,``soft" decisions are made meaning that the heuristic would begin with the decisions from the previous solution. As scans pass beyond the horizon and out of the sliding window, the decisions become fixed and we refer to them as ``hard" decisions. This process continues until all scans have been processed. The run times of a sliding window variant of the heuristic would not exhibit the curse of dimensionality in $T$ since the number of scans remains constant. Additionally, the heuristic is likely to produce higher quality solutions as a result of these ``soft" decisions of previous steps since it is starting from a solution which is likely to be better than a completely random solution.  

We additionally saw that one of the key challenges in the case of detection ambiguity is to correctly estimate the true number of targets. This is done here by modifying the penalties for missed detection and false alarms. Although we succeed in doing so for many cases, we observed that scenarios with different number of targets may need different penalties. This implies that there is a need for additional research which explores the use of more complex penalty functions, which implicitly take into account this dependency . %One possible idea would be the use of piecewise linear functions which would require further expansion of our formulations. 
Other areas for further research may lie in relaxing some of the scenario based assumptions, in particular, extensions to non-linear trajectories or the birth/death of targets. 