In this paper, we present a new approach to the multi-target tracking problem that jointly solves the problems of data association and trajectory estimation via global optimization methods using a single objective function. In order to solve these problems efficiently we propose the use of a randomized local search heuristic as a warm start for an MIO model, while addressing scenarios with and without detection ambiguity. The described approach is both general, because it makes no assumptions on the data generation process, and easily implementable, as it is based on a simple model with little to no tunable parameters. We demonstrate that the local search heuristic finds high quality feasible solutions very quickly, and that the MIO model generates improvement over the heuristic in seconds. Furthermore, we show that while the introduction of detection ambiguity deteriorates the performance of the data association, it does not significantly effect the quality of the trajectory estimation. We are able to conclude that the quality of the  obtained solution is robust to the missed detection rate, but less so to the false alarm rate.

In the process of analyzing this widely-studied problem through a new optimization lens, we identified challenges in both model formulation and successful implementation that can be addressed in future work. Due to the fact that the runtimes of the MIO and the heuristic are proportional to the number of scans, these models have limited scalability in that sense. However, they show strong potential for larger applications in a sliding window scheme, which would use past decisions to fix detection assignments and thus contribute additional information to the current tracking window. This would allow for the tracking of targets in real time systems for longer periods of observation. Additionally, we observed that one of the key difficulties in the case of detection ambiguity involves the correct estimation of the true number of targets. Our results suggest that tuning the penalties for a scenario with a certain number of targets may lead to over or underestimation for scenarios with differing numbers of targets. Therefore, we suggest exploration into more complex penalties that further analyze this dependency. Other directions for consideration also lie in relaxing some of the scenario based assumptions, in particular, extensions to non-linear trajectories or the birth/death of targets. 