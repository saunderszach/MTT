In this paper, we presented a new approach to the multi-target tracking problem which jointly solves the problems of data association and trajectory estimation via global optimization methods using a single objective function. Toward this goal, we proposed the use of a randomized local search heuristic as a warm start for a mixed integer optimization model, and we did so for scenarios with and without detection ambiguity. The approach is general since it makes no assumptions on the data generation process, and easily implementable, since it is based on a simple model with little to no tunable parameters. We showed that the local search heuristic finds good quality feasible solutions very quickly, and that the MIO offers improvement over the heuristic in seconds. We also show that while the introduction of detection ambiguity deteriorates the performance of the data association, it does not significantly effect the quality of trajectory estimation. Moreover, the quality of the solution obtained is robust to the missed detection rate, but less so to the false alarm rate.

The MIO and the heuristic runtimes are proportional to the number of scans and therefore, have limited scalability in that sense. However, they show potential for use in a sliding window scheme, which would use past decisions to fix detection assignments and thus add information for the current tracking window. This would allow us to track targets in real time systems for longer periods. 

Additionally, we observed that one of the key challenges in the case of detection ambiguity involves the correct estimation of the true number of targets. Our results suggest that tuning the penalties for a scenario with a certain number of targets may lead to over or underestimation for scenarios with a different number of targets. Therefore, we suggest additional research into more complex penalties that explore this dependency. 

Other areas for further research may lie in relaxing some of the scenario based assumptions, in particular, extensions to non-linear trajectories or the birth/death of targets. 