For completeness here we present an alternative approach to solving the MTT problem with detection ambiguity. This MIO model is an extension to \eqref{eq:simple_robust} that directly determines the number of targets via optimization by incorporating additional variables and constraints into framework of the formulation. 

\mysubsection{Decision Variables}
Because we do not assume a fixed number of targets, this decision must now be made by the model. Toward this goal, we introduce a new binary decision variable $w_{j}$ to indicate whether or not trajectory \textit{j} corresponds to an existing target. Any detections assigned to a non-existing target would be a false alarm. 
\[w_{j} = 
\begin{cases}
1, & \text{if trajectory \textit{j} exists,}\\
0, & \text{otherwise.}
\end{cases}\]
Since we by Assumption~\ref{ass:robust_assumptions} the number of targets will be at most $N_1$, the number of needed variables will be $N_1$.

\mysubsection{Constraints}
Most constraints remain similar to their original counterparts in \eqref{eq:simple_robust}, except for slight adjustments needed to account for the possibility that some trajectories may not exist. Explicitly, the number of possible trajectories is now $N_{1}$ so all instances of summing over \textit{j} must be changed accordingly. For example, we adjust \eqref{eqn: FA Simple} and \eqref{eqn: MD Total} as follows: 
\begin{align*}
\sum_{j=1}^{N_{1}} y_{itj} + F_{it} = 1 \qquad \forall i,t,\\
\sum_{j=1}^{N_{1}} \sum_{t=1}^{T} M_{jt} = TM.
\end{align*}

By the same accord the RHS of \eqref{eqn: MD Simple} no longer equals to 1 because some trajectories may not exist. Therefore we say all \textit{existing} trajectories must either be assigned a detection or a missed detection, which implies the following constraints
\begin{align}\label{eqn: Existing Targets}
\sum_{i=1}^{n_{t}} y_{itj} + M_{jt} = w_{j} \qquad \forall j,t.
\end{align}

Moreover, we restrict $\alpha_{j}$ and $\beta_{j}$ to be zero if trajectory \textit{j} does not exist. This ensures only existing trajectories are penalized in the objective function. 
\begin{align*}
|\alpha_{j}|+|\beta_{j}| \leq M_{0}w_{j}\qquad \forall j.
\end{align*}

This model is symmetric between various $w_j$ variables. Such symmetry in models is an undesirable quality since the multiple optimal solutions cause solving such problems to be less efficient. We can actually reduce the symmetry that is inherent to this formulation. Since $N_{0} \leq P \leq N_{1}$, we can set $w_j=1$ for all $j=1,\ldots,N_0$, which leaves us with only $N_1-N_0$ additional binary variables. Adding the constraints
\begin{align*}
w_{N_0+1}\geq ...\geq w_{N_1},
\end{align*}
further reduces the number of equivalent solution and increases the efficient resolvability of the model. 

\mysubsection{Full Formulation}
Incorporating these additional variables and constraints, we arrive at the following complete alternative formulation.
\begin{align*}
\underset{\psi_{jt}}{\text{minimize: }} & \sum_{j=1}^{N_{1}} \sum_{t=1}^{T} \psi_{jt} + \theta \cdot TF + \phi \cdot TM\\
\text{subject to: }	& \sum_{j=1}^{N_{1}} y_{itj} + F_{it} = 1 \qquad \forall i,t \nonumber\\
				& \sum_{i=1}^{n_{t}} y_{itj} + M_{jt} = 1 \qquad \forall j=1,...,N_{0},t \nonumber \\
				& \sum_{i=1}^{n_{t}} y_{itj} + M_{jt} = w_{j} \qquad \forall j=N_{0},...,N_{1},t \nonumber \\
				& \sum_{i=1}^{n_{t}} \sum_{t=1}^{T} F_{it} = TF \nonumber \\
				& \sum_{j=1}^{N_{1}} \sum_{t=1}^{T} M_{jt} = TM \nonumber \\
				& w_{N_0+1}\geq ...\geq w_{N_1} \nonumber \\
				& |\alpha_{j}|+|\beta_{j}| \leq M_{0}w_{j}\qquad \forall j \nonumber \\
				& x_{it}y_{itj} + M_{t}(1-y_{itj}) \geq z_{jt} \qquad \forall i,t,j \nonumber \\
				& x_{it}y_{itj} - M_{t}(1-y_{itj}) \leq z_{jt} \qquad \forall i,t,j \nonumber \\
				& z_{jt} - \alpha_{j} - \beta_{j}t \leq \psi_{jt} \qquad \forall j,t \nonumber \\
				& -(z_{jt} - \alpha_{j} - \beta_{j}t) \leq \psi_{jt} \qquad \forall j,t \nonumber \\
			 	& y_{itj} \in \{0,1\} \quad \forall i,t,j \nonumber \\
				& \alpha_{j} \in \mathbb{R}^n,\quad \beta_{j} \in \mathbb{R}^n,\quad w_{j} \in \mathbb{R}^n \quad \forall j \nonumber \\
				& z_{jt} \in \mathbb{R}^n, \quad \forall j,t. \nonumber
\end{align*}

\mysubsection{Extension of Robust Heuristic}
In order to initialize the above MIO we need to explore the entire state space which include specifying the number of targets. Such a heuritic will be identical to that proposed in \mysection~\ref{\myabrv Robust MIO Model}, except the number of target used will also be randomly  selected during the initialization process. As in all previous cases, the warm start will be chosen among the different heuristic solutions based on the objective function value.