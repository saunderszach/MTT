Here we provide recommendations for the tuning of penalty parameters $\theta$ and $\phi$. We begin with an explanation of grounded in logic. It can be shown that as the false alarm rate $\lambda$ increases, the number of expected false alarms also increases. Therefore, it stands to reason that as a general rule of thumb the false alarm penalty $\theta$ should decrease as $\lambda$ increases. Similarly, the number of expected missed detections increases as the missed detection probability $\gamma$ increases, and so too the missed detection penalty should decrease. Then it follows logically that the missed detection penalty $\phi$ should increase as $\gamma$ decreases. Furthermore, it is convenient to reason that the value of both of these penalties should somehow be tied to the value of $\sigma$, though this is a more difficult sequence of logic to justify. Through examination we found these logical concepts to hold true across a variety of scenario sizes and difficulties. Using this insight as well as the results of a mini experiment, we arrived at the penalty values summarized in Table~\ref{tab:Penalties}.


%%% IEEE Paper %%%
\begin{table}[ht]
\centering
\begin{tabular}{c|c|c|c|c}
\hline
$\sigma$ & $\gamma$ & $\lambda$ & $\theta$ & $\phi$\\
\hline

%%% Thesis %%%
%\begin{longtable}{c|c|c|c|c}
%\hline
%$\sigma$ & $\gamma$ & $\lambda$ & $\theta$ & $\phi$\\
%\hline
%\endhead

0.1 & 0.2 & 0.1 & 0.4 & 0.1 \\
0.1 & 0.2 & 0.5 & 0.2 & 0.1 \\
0.1 & 0.2 & 1.0 & 0.3 & 0.2 \\
0.1 & 0.2 & 2.0 & 0.1 & 0.2 \\
0.1 & 0.15 & 0.1 & 0.4 & 0.1 \\
0.1 & 0.15 & 0.5 & 0.4 & 0.1 \\
0.1 & 0.15 & 1.0 & 0.4 & 0.3 \\
0.1 & 0.15 & 2.0 & 0.1 & 0.3 \\
0.1 & 0.1 & 0.1 & 0.3 & 0.1 \\
0.1 & 0.1 & 0.5 & 0.3 & 0.2 \\
0.1 & 0.1 & 1.0 & 0.2 & 0.2 \\
0.1 & 0.1 & 2.0 & 0.1 & 0.3 \\
0.1 & 0.05 & 0.1 & 0.3 & 0.2 \\
0.1 & 0.05 & 0.5 & 0.2 & 0.2 \\
0.1 & 0.05 & 1.0 & 0.1 & 0.4 \\
0.1 & 0.05 & 2.0 & 0.1 & 0.4 \\
0.5 & 0.2 & 0.1 & 0.5 & 0.1 \\
0.5 & 0.2 & 0.5 & 0.4 & 0.2 \\
0.5 & 0.2 & 1.0 & 0.4 & 0.2 \\
0.5 & 0.2 & 2.0 & 0.3 & 0.3 \\
0.5 & 0.15 & 0.1 & 0.5 & 0.1 \\
0.5 & 0.15 & 0.5 & 0.4 & 0.2 \\
0.5 & 0.15 & 1.0 & 0.4 & 0.3 \\
0.5 & 0.15 & 2.0 & 0.3 & 0.4 \\
0.5 & 0.1 & 0.1 & 0.4 & 0.3 \\
0.5 & 0.1 & 0.5 & 0.4 & 0.3 \\
0.5 & 0.1 & 1.0 & 0.3 & 0.3 \\
0.5 & 0.1 & 2.0 & 0.2 & 0.4 \\
0.5 & 0.05 & 0.1 & 0.4 & 0.4 \\
0.5 & 0.05 & 0.5 & 0.4 & 0.4 \\
0.5 & 0.05 & 1.0 & 0.3 & 0.5 \\
0.5 & 0.05 & 2.0 & 0.3 & 0.5 \\
1.0 & 0.2 & 0.1 & 0.5 & 0.1 \\
1.0 & 0.2 & 0.5 & 0.5 & 0.2 \\
1.0 & 0.2 & 1.0 & 0.5 & 0.1 \\
1.0 & 0.2 & 2.0 & 0.5 & 0.4 \\
1.0 & 0.15 & 0.1 & 0.5 & 0.1 \\
1.0 & 0.15 & 0.5 & 0.5 & 0.3 \\
1.0 & 0.15 & 1.0 & 0.5 & 0.2 \\
1.0 & 0.15 & 2.0 & 0.5 & 0.4 \\
1.0 & 0.1 & 0.1 & 0.5 & 0.5 \\
1.0 & 0.1 & 0.5 & 0.5 & 0.4 \\
1.0 & 0.1 & 1.0 & 0.4 & 0.4 \\
1.0 & 0.1 & 2.0 & 0.4 & 0.5 \\
1.0 & 0.05 & 0.1 & 0.5 & 0.5 \\
1.0 & 0.05 & 0.5 & 0.5 & 0.5 \\
1.0 & 0.05 & 1.0 & 0.5 & 0.5 \\
1.0 & 0.05 & 2.0 & 0.5 & 0.5 \\
2.0 & 0.2 & 0.1 & 0.5 & 0.1 \\
2.0 & 0.2 & 0.5 & 0.5 & 0.2 \\
2.0 & 0.2 & 1.0 & 0.5 & 0.1 \\
2.0 & 0.2 & 2.0 & 0.5 & 0.5 \\
2.0 & 0.15 & 0.1 & 0.5 & 0.1 \\
2.0 & 0.15 & 0.5 & 0.5 & 0.3 \\
2.0 & 0.15 & 1.0 & 0.5 & 0.2 \\
2.0 & 0.15 & 2.0 & 0.5 & 0.5 \\
2.0 & 0.1 & 0.1 & 0.5 & 0.5 \\
2.0 & 0.1 & 0.5 & 0.5 & 0.5 \\
2.0 & 0.1 & 1.0 & 0.5 & 0.5 \\
2.0 & 0.1 & 2.0 & 0.5 & 0.5 \\
2.0 & 0.05 & 0.1 & 0.5 & 0.5 \\
2.0 & 0.05 & 0.5 & 0.5 & 0.5 \\
2.0 & 0.05 & 1.0 & 0.5 & 0.5 \\
2.0 & 0.05 & 2.0 & 0.5 & 0.5 \\
\hline

%%% IEEE %%%
\end{tabular}
\caption{Experiment 2 penalty values.}
\label{tab:Penalties}
\end{table}

%%% Thesis %%%
%\caption{Experiment 2 penalty values.}
%\label{tab:Penalties}
%\end{longtable}



