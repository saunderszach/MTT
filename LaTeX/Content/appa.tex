Here we provide recommendations for the tuning of penalty parameters $\theta$ and $\phi$. We begin with an explanation grounded in logic. It can be shown that as the false alarm rate $\lambda$ increases, the number of expected false alarms also increases. Therefore, it stands to reason that as a general rule of thumb the false alarm penalty $\theta$ should decrease as $\lambda$ increases. Similarly, the number of expected missed detections increases as the missed detection probability $\gamma$ increases, and so too the missed detection penalty should decrease. Then it follows logically that the missed detection penalty $\phi$ should increase as $\gamma$ decreases. Furthermore, it is convenient to reason that the value of both of these penalties should somehow be tied to the value of $\sigma$. This is due to the fact that as the noise increases, expect a higher standard deviation for the detections. 
Thus, when assigning false alarms and missed detections, then error will naturally be higher in the objective function. To combat this effect, we should consider increasing both penalties as $\sigma$ increases. Through examination we found these logical concepts to generally hold true across a variety of scenario sizes and difficulties. Using this insight as well as the results of a mini experiment, we tuned our penalties for the robust experiment. The false alarm penalties used are shown in Figure~\ref{tab:Theta_Penalties} and the missed detection penalties are shown in Figure~\ref{tab:Phi_Penalties}.
\begin{table}[ht]
\centering
\begin{tabular}{c|m{1cm}m{1cm}m{1cm}m{1cm}}
  \hline
   & \multicolumn{4}{c}{$\sigma$} \\
   \cline{2-5}
   $\lambda$ & 0.1 & 0.5 & 1.0 & 2.0\\
  \hline 
   0.1 & 1.7 & 2.6 & 3.1 & 3.5 \\
   0.5 & 1.1 & 1.9 & 2.3 & 2.5 \\ 
   1.0 & 0.9 & 1.2 & 1.6 & 1.8 \\ 
   2.0 & 0.5 & 0.9 & 0.9 & 1.0 \\ 
   \hline
\end{tabular}
\caption{False alarm penalties ($\theta$) as a function of $\lambda$ and $\sigma$.}
\label{tab:Theta_Penalties}
\end{table}
\begin{table}[ht]
\centering
\begin{tabular}{cc|cccc}
  \hline
  & & \multicolumn{4}{c}{$\sigma$} \\
  \cline{3-6}
 $\lambda$ & $\gamma$ & 0.1 & 0.5 & 1 & 2 \\ 
  \hline
   0.10 & 0.05 & 0.20 & 0.50 & 0.80 & 0.70 \\ 
   0.10 & 0.10 & 0.10 & 0.30 & 0.50 & 0.50 \\ 
   0.10 & 0.15 & 0.10 & 0.20 & 0.40 & 0.40 \\ 
   0.10 & 0.20 & 0.10 & 0.10 & 0.30 & 0.40 \\ 
   0.50 & 0.05 & 0.20 & 0.50 & 0.80 & 0.80 \\ 
   0.50 & 0.10 & 0.20 & 0.30 & 0.50 & 0.60 \\ 
   0.50 & 0.15 & 0.20 & 0.25 & 0.40 & 0.40 \\ 
   0.50 & 0.20 & 0.10 & 0.20 & 0.30 & 0.40 \\ 
   1.00 & 0.05 & 0.30 & 0.70 & 0.80 & 0.80 \\ 
   1.00 & 0.10 & 0.20 & 0.40 & 0.50 & 0.60 \\ 
   1.00 & 0.15 & 0.20 & 0.25 & 0.40 & 0.40 \\ 
   1.00 & 0.20 & 0.10 & 0.20 & 0.30 & 0.40 \\ 
   2.00 & 0.05 & 0.30 & 0.70 & 0.90 & 1.00 \\ 
   2.00 & 0.10 & 0.20 & 0.50 & 0.60 & 0.60 \\ 
   2.00 & 0.15 & 0.20 & 0.25 & 0.40 & 0.50 \\ 
   2.00 & 0.20 & 0.10 & 0.20 & 0.30 & 0.40 \\ 
   \hline
\end{tabular}
\caption{Missed detection penalties ($\phi$) as a function of $\lambda$, $\gamma$, and $\sigma$.}
\label{tab:Phi_Penalties}
\end{table}


%%%% IEEE Paper %%%
%\begin{table}[ht]
%\centering
%\begin{tabular}{c|c|c|c|c}
%\hline
%$\sigma$ & $\gamma$ & $\lambda$ & $\theta$ & $\phi$\\
%\hline
%0.1 & 0.2 & 0.1 & 0.4 & 0.1 \\
%0.1 & 0.2 & 0.5 & 0.2 & 0.1 \\
%0.1 & 0.2 & 1.0 & 0.3 & 0.2 \\
%0.1 & 0.2 & 2.0 & 0.1 & 0.2 \\
%0.1 & 0.15 & 0.1 & 0.4 & 0.1 \\
%0.1 & 0.15 & 0.5 & 0.4 & 0.1 \\
%0.1 & 0.15 & 1.0 & 0.4 & 0.3 \\
%0.1 & 0.15 & 2.0 & 0.1 & 0.3 \\
%0.1 & 0.1 & 0.1 & 0.3 & 0.1 \\
%0.1 & 0.1 & 0.5 & 0.3 & 0.2 \\
%0.1 & 0.1 & 1.0 & 0.2 & 0.2 \\
%0.1 & 0.1 & 2.0 & 0.1 & 0.3 \\
%0.1 & 0.05 & 0.1 & 0.3 & 0.2 \\
%0.1 & 0.05 & 0.5 & 0.2 & 0.2 \\
%0.1 & 0.05 & 1.0 & 0.1 & 0.4 \\
%0.1 & 0.05 & 2.0 & 0.1 & 0.4 \\
%0.5 & 0.2 & 0.1 & 0.5 & 0.1 \\
%0.5 & 0.2 & 0.5 & 0.4 & 0.2 \\
%0.5 & 0.2 & 1.0 & 0.4 & 0.2 \\
%0.5 & 0.2 & 2.0 & 0.3 & 0.3 \\
%0.5 & 0.15 & 0.1 & 0.5 & 0.1 \\
%0.5 & 0.15 & 0.5 & 0.4 & 0.2 \\
%0.5 & 0.15 & 1.0 & 0.4 & 0.3 \\
%0.5 & 0.15 & 2.0 & 0.3 & 0.4 \\
%0.5 & 0.1 & 0.1 & 0.4 & 0.3 \\
%0.5 & 0.1 & 0.5 & 0.4 & 0.3 \\
%0.5 & 0.1 & 1.0 & 0.3 & 0.3 \\
%0.5 & 0.1 & 2.0 & 0.2 & 0.4 \\
%0.5 & 0.05 & 0.1 & 0.4 & 0.4 \\
%0.5 & 0.05 & 0.5 & 0.4 & 0.4 \\
%0.5 & 0.05 & 1.0 & 0.3 & 0.5 \\
%0.5 & 0.05 & 2.0 & 0.3 & 0.5 \\
%1.0 & 0.2 & 0.1 & 0.5 & 0.1 \\
%1.0 & 0.2 & 0.5 & 0.5 & 0.2 \\
%1.0 & 0.2 & 1.0 & 0.5 & 0.1 \\
%1.0 & 0.2 & 2.0 & 0.5 & 0.4 \\
%1.0 & 0.15 & 0.1 & 0.5 & 0.1 \\
%1.0 & 0.15 & 0.5 & 0.5 & 0.3 \\
%1.0 & 0.15 & 1.0 & 0.5 & 0.2 \\
%1.0 & 0.15 & 2.0 & 0.5 & 0.4 \\
%1.0 & 0.1 & 0.1 & 0.5 & 0.5 \\
%1.0 & 0.1 & 0.5 & 0.5 & 0.4 \\
%1.0 & 0.1 & 1.0 & 0.4 & 0.4 \\
%1.0 & 0.1 & 2.0 & 0.4 & 0.5 \\
%1.0 & 0.05 & 0.1 & 0.5 & 0.5 \\
%1.0 & 0.05 & 0.5 & 0.5 & 0.5 \\
%1.0 & 0.05 & 1.0 & 0.5 & 0.5 \\
%1.0 & 0.05 & 2.0 & 0.5 & 0.5 \\
%2.0 & 0.2 & 0.1 & 0.5 & 0.1 \\
%2.0 & 0.2 & 0.5 & 0.5 & 0.2 \\
%2.0 & 0.2 & 1.0 & 0.5 & 0.1 \\
%2.0 & 0.2 & 2.0 & 0.5 & 0.5 \\
%2.0 & 0.15 & 0.1 & 0.5 & 0.1 \\
%2.0 & 0.15 & 0.5 & 0.5 & 0.3 \\
%2.0 & 0.15 & 1.0 & 0.5 & 0.2 \\
%2.0 & 0.15 & 2.0 & 0.5 & 0.5 \\
%2.0 & 0.1 & 0.1 & 0.5 & 0.5 \\
%2.0 & 0.1 & 0.5 & 0.5 & 0.5 \\
%2.0 & 0.1 & 1.0 & 0.5 & 0.5 \\
%2.0 & 0.1 & 2.0 & 0.5 & 0.5 \\
%2.0 & 0.05 & 0.1 & 0.5 & 0.5 \\
%2.0 & 0.05 & 0.5 & 0.5 & 0.5 \\
%2.0 & 0.05 & 1.0 & 0.5 & 0.5 \\
%2.0 & 0.05 & 2.0 & 0.5 & 0.5 \\
%\hline
%\end{tabular}
%\caption{Experiment 2 penalty values.}
%\label{tab:Penalties}
%\end{table}
%
%
%%%% Thesis %%%
%\begin{longtable}{c|c|c|c|c}
%\hline
%$\sigma$ & $\gamma$ & $\lambda$ & $\theta$ & $\phi$\\
%\hline
%\endhead
%0.1 & 0.2 & 0.1 & 0.4 & 0.1 \\
%0.1 & 0.2 & 0.5 & 0.2 & 0.1 \\
%0.1 & 0.2 & 1.0 & 0.3 & 0.2 \\
%0.1 & 0.2 & 2.0 & 0.1 & 0.2 \\
%0.1 & 0.15 & 0.1 & 0.4 & 0.1 \\
%0.1 & 0.15 & 0.5 & 0.4 & 0.1 \\
%0.1 & 0.15 & 1.0 & 0.4 & 0.3 \\
%0.1 & 0.15 & 2.0 & 0.1 & 0.3 \\
%0.1 & 0.1 & 0.1 & 0.3 & 0.1 \\
%0.1 & 0.1 & 0.5 & 0.3 & 0.2 \\
%0.1 & 0.1 & 1.0 & 0.2 & 0.2 \\
%0.1 & 0.1 & 2.0 & 0.1 & 0.3 \\
%0.1 & 0.05 & 0.1 & 0.3 & 0.2 \\
%0.1 & 0.05 & 0.5 & 0.2 & 0.2 \\
%0.1 & 0.05 & 1.0 & 0.1 & 0.4 \\
%0.1 & 0.05 & 2.0 & 0.1 & 0.4 \\
%0.5 & 0.2 & 0.1 & 0.5 & 0.1 \\
%0.5 & 0.2 & 0.5 & 0.4 & 0.2 \\
%0.5 & 0.2 & 1.0 & 0.4 & 0.2 \\
%0.5 & 0.2 & 2.0 & 0.3 & 0.3 \\
%0.5 & 0.15 & 0.1 & 0.5 & 0.1 \\
%0.5 & 0.15 & 0.5 & 0.4 & 0.2 \\
%0.5 & 0.15 & 1.0 & 0.4 & 0.3 \\
%0.5 & 0.15 & 2.0 & 0.3 & 0.4 \\
%0.5 & 0.1 & 0.1 & 0.4 & 0.3 \\
%0.5 & 0.1 & 0.5 & 0.4 & 0.3 \\
%0.5 & 0.1 & 1.0 & 0.3 & 0.3 \\
%0.5 & 0.1 & 2.0 & 0.2 & 0.4 \\
%0.5 & 0.05 & 0.1 & 0.4 & 0.4 \\
%0.5 & 0.05 & 0.5 & 0.4 & 0.4 \\
%0.5 & 0.05 & 1.0 & 0.3 & 0.5 \\
%0.5 & 0.05 & 2.0 & 0.3 & 0.5 \\
%1.0 & 0.2 & 0.1 & 0.5 & 0.1 \\
%1.0 & 0.2 & 0.5 & 0.5 & 0.2 \\
%1.0 & 0.2 & 1.0 & 0.5 & 0.1 \\
%1.0 & 0.2 & 2.0 & 0.5 & 0.4 \\
%1.0 & 0.15 & 0.1 & 0.5 & 0.1 \\
%1.0 & 0.15 & 0.5 & 0.5 & 0.3 \\
%1.0 & 0.15 & 1.0 & 0.5 & 0.2 \\
%1.0 & 0.15 & 2.0 & 0.5 & 0.4 \\
%1.0 & 0.1 & 0.1 & 0.5 & 0.5 \\
%1.0 & 0.1 & 0.5 & 0.5 & 0.4 \\
%1.0 & 0.1 & 1.0 & 0.4 & 0.4 \\
%1.0 & 0.1 & 2.0 & 0.4 & 0.5 \\
%1.0 & 0.05 & 0.1 & 0.5 & 0.5 \\
%1.0 & 0.05 & 0.5 & 0.5 & 0.5 \\
%1.0 & 0.05 & 1.0 & 0.5 & 0.5 \\
%1.0 & 0.05 & 2.0 & 0.5 & 0.5 \\
%2.0 & 0.2 & 0.1 & 0.5 & 0.1 \\
%2.0 & 0.2 & 0.5 & 0.5 & 0.2 \\
%2.0 & 0.2 & 1.0 & 0.5 & 0.1 \\
%2.0 & 0.2 & 2.0 & 0.5 & 0.5 \\
%2.0 & 0.15 & 0.1 & 0.5 & 0.1 \\
%2.0 & 0.15 & 0.5 & 0.5 & 0.3 \\
%2.0 & 0.15 & 1.0 & 0.5 & 0.2 \\
%2.0 & 0.15 & 2.0 & 0.5 & 0.5 \\
%2.0 & 0.1 & 0.1 & 0.5 & 0.5 \\
%2.0 & 0.1 & 0.5 & 0.5 & 0.5 \\
%2.0 & 0.1 & 1.0 & 0.5 & 0.5 \\
%2.0 & 0.1 & 2.0 & 0.5 & 0.5 \\
%2.0 & 0.05 & 0.1 & 0.5 & 0.5 \\
%2.0 & 0.05 & 0.5 & 0.5 & 0.5 \\
%2.0 & 0.05 & 1.0 & 0.5 & 0.5 \\
%2.0 & 0.05 & 2.0 & 0.5 & 0.5 \\
%\hline
%\caption{Experiment 2 penalty values.}
%\label{tab:Penalties}
%\end{longtable}