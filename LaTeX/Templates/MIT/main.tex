% -*- Mode:TeX -*-

%% IMPORTANT: The official thesis specifications are available at:
%%            http://libraries.mit.edu/archives/thesis-specs/
%%
%%            Please verify your thesis' formatting and copyright
%%            assignment before submission.  If you notice any
%%            discrepancies between these templates and the 
%%            MIT Libraries' specs, please let us know
%%            by e-mailing thesis@mit.edu

%% The documentclass options along with the pagestyle can be used to generate
%% a technical report, a draft copy, or a regular thesis.  You may need to
%% re-specify the pagestyle after you \include  cover.tex.  For more
%% information, see the first few lines of mitthesis.cls. 

%\documentclass[12pt,vi,twoside]{mitthesis}
%%
%%  If you want your thesis copyright to you instead of MIT, use the
%%  ``vi'' option, as above.
%%
%\documentclass[12pt,twoside,leftblank]{mitthesis}
%%
%% If you want blank pages before new chapters to be labelled ``This
%% Page Intentionally Left Blank'', use the ``leftblank'' option, as
%% above. 

\documentclass[12pt,twoside]{mitthesis}
\usepackage{lgrind}
%% These have been added at the request of the MIT Libraries, because
%% some PDF conversions mess up the ligatures.  -LB, 1/22/2014
\usepackage{cmap}
\usepackage[T1]{fontenc}
\pagestyle{plain}

%% This bit allows you to either specify only the files which you wish to
%% process, or `all' to process all files which you \include.
%% Krishna Sethuraman (1990).

\typein [\files]{Enter file names to process, (chap1,chap2 ...), or `all' to
process all files:}
\def\all{all}
\ifx\files\all \typeout{Including all files.} \else \typeout{Including only \files.} \includeonly{\files} \fi

\begin{document}

% -*-latex-*-
% 
% For questions, comments, concerns or complaints:
% thesis@mit.edu
% 
%
% $Log: cover.tex,v $
% Revision 1.8  2008/05/13 15:02:15  jdreed
% Degree month is June, not May.  Added note about prevdegrees.
% Arthur Smith's title updated
%
% Revision 1.7  2001/02/08 18:53:16  boojum
% changed some \newpages to \cleardoublepages
%
% Revision 1.6  1999/10/21 14:49:31  boojum
% changed comment referring to documentstyle
%
% Revision 1.5  1999/10/21 14:39:04  boojum
% *** empty log message ***
%
% Revision 1.4  1997/04/18  17:54:10  othomas
% added page numbers on abstract and cover, and made 1 abstract
% page the default rather than 2.  (anne hunter tells me this
% is the new institute standard.)
%
% Revision 1.4  1997/04/18  17:54:10  othomas
% added page numbers on abstract and cover, and made 1 abstract
% page the default rather than 2.  (anne hunter tells me this
% is the new institute standard.)
%
% Revision 1.3  93/05/17  17:06:29  starflt
% Added acknowledgements section (suggested by tompalka)
% 
% Revision 1.2  92/04/22  13:13:13  epeisach
% Fixes for 1991 course 6 requirements
% Phrase "and to grant others the right to do so" has been added to 
% permission clause
% Second copy of abstract is not counted as separate pages so numbering works
% out
% 
% Revision 1.1  92/04/22  13:08:20  epeisach

% NOTE:
% These templates make an effort to conform to the MIT Thesis specifications,
% however the specifications can change.  We recommend that you verify the
% layout of your title page with your thesis advisor and/or the MIT 
% Libraries before printing your final copy.
\title{Multi-target Tracking via Mixed Integer Optimization}

\author{Zachary Clayton Saunders}
% If you wish to list your previous degrees on the cover page, use the 
% previous degrees command:
       \prevdegrees{B.S. Operations Research, United States Air Force Academy (2014)}
% You can use the \\ command to list multiple previous degrees
%       \prevdegrees{B.S., University of California (1978) \\
%                    S.M., Massachusetts Institute of Technology (1981)}
\department{Sloan School of Management}

% If the thesis is for two degrees simultaneously, list them both
% separated by \and like this:
% \degree{Doctor of Philosophy \and Master of Science}
\degree{Master of Science in Operations Research}

% As of the 2007-08 academic year, valid degree months are September, 
% February, or June.  The default is June.
\degreemonth{June}
\degreeyear{2016}
\thesisdate{May 13, 2016}

%% By default, the thesis will be copyrighted to MIT.  If you need to copyright
%% the thesis to yourself, just specify the `vi' documentclass option.  If for
%% some reason you want to exactly specify the copyright notice text, you can
%% use the \copyrightnoticetext command.  
%\copyrightnoticetext{\copyright IBM, 1990.  Do not open till Xmas.}

% If there is more than one supervisor, use the \supervisor command
% once for each.
\supervisor{Sung-Hyun Son}{Assistant Group Leader, Lincoln Laboratory Group 36}
\supervisor{Dimitris Bertsimas}{Boeing Professor of Operations Research\\
						Co-Director, Operations Research Center}

% This is the department committee chairman, not the thesis committee
% chairman.  You should replace this with your Department's Committee
% Chairman.
\chairman{Patrick Jaillet}{Dugald C. Jackson Professor\\
				       Department of Electrical Engineering and Computer Science\\
				       Co-Director, Operations Research Center}

% Make the titlepage based on the above information.  If you need
% something special and can't use the standard form, you can specify
% the exact text of the titlepage yourself.  Put it in a titlepage
% environment and leave blank lines where you want vertical space.
% The spaces will be adjusted to fill the entire page.  The dotted
% lines for the signatures are made with the \signature command.
\maketitle

% The abstractpage environment sets up everything on the page except
% the text itself.  The title and other header material are put at the
% top of the page, and the supervisors are listed at the bottom.  A
% new page is begun both before and after.  Of course, an abstract may
% be more than one page itself.  If you need more control over the
% format of the page, you can use the abstract environment, which puts
% the word "Abstract" at the beginning and single spaces its text.

%% You can either \input (*not* \include) your abstract file, or you can put
%% the text of the abstract directly between the \begin{abstractpage} and
%% \end{abstractpage} commands.

% First copy: start a new page, and save the page number.
\cleardoublepage
% Uncomment the next line if you do NOT want a page number on your
% abstract and acknowledgments pages.
% \pagestyle{empty}
\setcounter{savepage}{\thepage}
\begin{abstractpage}
The field of multi-target tracking (MTT) faces two primary challenges: (i) data association and (ii) trajectory estimation. Many algorithms, such as the MHT and JPDAF, attempt to solve the MTT problem via probabilistic approaches, while few employ the use of optimization models. In this paper we take a new approach to the MTT problem by modeling it as a mixed integer optimization (MIO) model, making no probabilistic assumptions on the data. We model the basic case where there is no detection ambiguity, \textit{i.e.}, no false alarms or missed detections, and extend it to the more difficult robust case where there exists detection ambiguity. These models solve the data association and trajectory estimation problems simultaneously by minimizing an easily interpretable global objective function. Additionally, we propose a greedy heuristic that scales efficiently and finds high quality solutions in milliseconds, and we show that these heuristic solutions can be used effectively as a warm start for the MIO, enabling the MIO to achieve good solutions in seconds. In the basic case, the model has no tunable parameters and the extension to the ambiguous case requires only two easily understood parameters which serve as objective function penalties. Furthermore, we introduce a new metric for measuring complexity in the data association problem as well as a metric for measuring the performance of trajectory estimation. Numerical results show that in the case of the basic model we obtain a good reconstruction of the original tracks for a range of scenarios, while in the robust case, for small to medium sized scenarios we are able to both estimate the correct number of targets as well as correctly reconstruct the trajectories.
\end{abstractpage}

% Additional copy: start a new page, and reset the page number.  This way,
% the second copy of the abstract is not counted as separate pages.
% Uncomment the next 6 lines if you need two copies of the abstract
% page.
% \setcounter{page}{\thesavepage}
% \begin{abstractpage}
% The field of multi-target tracking (MTT) faces two primary challenges: (i) data association and (ii) trajectory estimation. Many algorithms, such as the MHT and JPDAF, attempt to solve the MTT problem via probabilistic approaches, while few employ the use of optimization models. In this paper we take a new approach to the MTT problem by modeling it as a mixed integer optimization (MIO) model, making no probabilistic assumptions on the data. We model the basic case where there is no detection ambiguity, \textit{i.e.}, no false alarms or missed detections, and extend it to the more difficult robust case where there exists detection ambiguity. These models solve the data association and trajectory estimation problems simultaneously by minimizing an easily interpretable global objective function. Additionally, we propose a greedy heuristic that scales efficiently and finds high quality solutions in milliseconds, and we show that these heuristic solutions can be used effectively as a warm start for the MIO, enabling the MIO to achieve good solutions in seconds. In the basic case, the model has no tunable parameters and the extension to the ambiguous case requires only two easily understood parameters which serve as objective function penalties. Furthermore, we introduce a new metric for measuring complexity in the data association problem as well as a metric for measuring the performance of trajectory estimation. Numerical results show that in the case of the basic model we obtain a good reconstruction of the original tracks for a range of scenarios, while in the robust case, for small to medium sized scenarios we are able to both estimate the correct number of targets as well as correctly reconstruct the trajectories.
% \end{abstractpage}

\cleardoublepage

\section*{Acknowledgments}
I would like to thank everyone who played a role in making this opportunity possible, everyone who supported me throughout this process, and everyone who influenced this project in any manner. Though I unfortunately do not have the space to thank each of you by name, I am immensely grateful for each and every one of you and the impacts you have had on me and this work. 

I would like to thank my advisor, Professor Dimitris Bertsimas, for his ongoing support and
guidance throughout this project. Without your direction and ideas none of this would have been possible. I thank you for constantly challenging me to push myself and for propelling me to expand my academic prowess beyond what I could have thought possible for myself. Additionally, I would like to thank Shimrit Shtern for her guidance and counseling on this project as well. Thank you for taking the time to pour over my error ridden scripts, edit this paper, and provide further guidance at critical points of this project. 

Thank you to everyone at Lincoln Laboratories who played a role in making this degree possible. To Mr John Kuconis, thank you for facilitating this opportunity and generously sponsoring me through a military fellowship. To my advisor, Sung-Hyun Son, I also want to thank you for providing me with the opportunity to be a Lincoln Laboratory Military Fellow in Group 36. Additionally, thank you for introducing me to the MTT problem and encouraging me to explore a field of the Air Force that was new to me, something that will no doubt pay dividends in my future as an officer. To Steve Relyea, thank you for meeting with me regularly and helping me weed through the details of the MTT problem, repeatedly scratching out ideas on paper. Your advice and insight was critical to both getting this project up and running, as well as keeping it on track throughout the process. Furthermore, I would like to thank the Lincoln Laboratories LLGrid team for their support in running my experiments. As a first time Linux user, your assistance was instrumental in running my simulations and gathering my results.

Finally, I want to thank my friends and family. To my family, Mom, Dad, Tess, Mathew, and Molly, thank you for your endless love and support and for constantly reminding me that hard work and due diligence always pays off. Without your continual support I would not be where I am today. I would also like to thank the truly extraordinary students at the MIT Operations Research Center who not only provided me with lasting friendships but also supported me academically both in the classroom and on this project. 
\vfill
CONTRACT ACKNOWLEDGMENT: This material is based upon work supported by the Air Force under Air Force Contract No. FA8721-05-C-0002 and/or FA8702-15-D-0001. The views expressed in this document are those of the author and do not reflect the official policy or position of the United States Air Force, the United States Department of Defense or the United States Government.

%%%%%%%%%%%%%%%%%%%%%%%%%%%%%%%%%%%%%%%%%%%%%%%%%%%%%%%%%%%%%%%%%%%%%%
% -*-latex-*-

% Some departments (e.g. 5) require an additional signature page.  See
% signature.tex for more information and uncomment the following line if
% applicable.
% % -*- Mode:TeX -*-
%
% Some departments (e.g. Chemistry) require an additional cover page
% with signatures of the thesis committee.  Please check with your
% thesis advisor or other appropriate person to determine if such a 
% page is required for your thesis.  
%
% If you choose not to use the "titlepage" environment, a \newpage
% commands, and several \vspace{\fill} commands may be necessary to
% achieve the required spacing.  The \signature command is defined in
% the "mitthesis" class
%
% The following sample appears courtesy of Ben Kaduk <kaduk@mit.edu> and
% was used in his June 2012 doctoral thesis in Chemistry. 

\begin{titlepage}
\begin{large}
This doctoral thesis has been examined by a Committee of the Department
of Chemistry as follows:

\signature{Professor Jianshu Cao}{Chairman, Thesis Committee \\
   Professor of Chemistry}

\signature{Professor Troy Van Voorhis}{Thesis Supervisor \\
   Associate Professor of Chemistry}

\signature{Professor Robert W. Field}{Member, Thesis Committee \\
   Haslam and Dewey Professor of Chemistry}
\end{large}
\end{titlepage}


\pagestyle{plain}
  % -*- Mode:TeX -*-
%% This file simply contains the commands that actually generate the table of
%% contents and lists of figures and tables.  You can omit any or all of
%% these files by simply taking out the appropriate command.  For more
%% information on these files, see appendix C.3.3 of the LaTeX manual. 
\tableofcontents
\newpage
\listoffigures
\newpage
\listoftables


Multi-target tracking is the problem of estimation the state of multiple dynamic objects, referred to as \textit{targets} over a fixed window of time. At various points of time within the window, the targets are observed in a \textit{scan}, resulting in set of \textit{detections}. From these detections, the multi-target tracking problem aims to extract information about target dynamics. 

Solutions to this problem are sought across many civilian and military applications including but not limited to ballistic missile and aircraft defense, space applications, the movement of ships and ground troops, autonomous vehicles and robotics, and air traffic control. Each application has unique attributes and assumptions, and various algorithms have been developed for each. As a result, the field of multi-target tracking has expanded to numerous research venues, and there is a wide range of literature on the topic. A complete overview of all MTT methods, including the classes of algorithms and their variants as well as additional methods not discussed in this paper, can be found in \cite{MTT-Taxonomy}. For a more exhaustive overview of estimation techniques, filtering, gating, and more please see \cite{Bar-Shalom_MTT} and \cite{Bar-Shalom_Estimation}.

The field of multi-target tracking faces two primary challenges: (i) data association and (ii) trajectory estimation.  Given a set of sensor detections, the data association problem consists of assigning the detections to a set of targets. Alternatively, this can be viewed as a labeling problem in which each detection needs to be labeled with a target identifier. The association problem is further complicated when sensors fail to report detections (missed detection) or incorrectly report detections (false alarm), resulting in ambiguity in the number of existing targets. The trajectory estimation problem consists of estimating the state space of a target (\textit{i.e.}, position, velocity, acceleration, size, etc.) from the associated detections of the aforementioned assignment problem. Even when all of the associations are known, the estimation problem is challenging due to the presence of measurement noise. As can be seen, the two problems of data association and trajectory estimation are closely related and dependent on one another. 

Some classical algorithms treat the data association and trajectory estimation problems separately using a combination of probabilistic approaches to determine data associations and filters to estimate trajectories. One such algorithm is the global nearest neighbor (GNN). The GNN algorithm is a naive 2-D assignment algorithm, which evaluates one scan of detections at a time, globally assigning the nearest detection at each scan \cite{GNN}. Once the data association has been determined, the detections are often passed through one of numerous filters, most commonly a Kalman filter \cite{Kalman}, which updates the trajectory estimates before the algorithm progresses forward to the next scan. This process repeats sequentially through each scan of data.

Modern algorithms in the field of multi-target tracking are most commonly statistical based, often relying on heavy probabilistic assumptions about the underlying target dynamics or detection process. The two most prevalent statistical algorithms in the field of multi-target tracking are the Multiple Hypothesis Tracker (MHT) and the Joint Probability Data Association Filter (JPDAF) and their numerous variants and extensions. Both classes of algorithms attempt to solve the data association problem by generating a set of potential hypotheses, or possible detection-to-track assignments. Here a \textit{track} is a set of labeled detections belonging to the same target. Probabilities are assigned to each hypothesis based on the likelihood of the trajectory's existence, and numerous approaches for accomplishing this task have been proposed.

The MHT, first proposed by Reid in \cite{MHT-Seminal}, assigns likelihood values to hypotheses using a Bayesian maximum a posteriori estimator, which requires probabilistic assumptions on both object dynamics and detection process. This algorithm is generally considered to be the modern standard for solving the data association problem. Many variants have been proposed for implementation which leverage techniques such as clustering, gating, hypothesis selection, hypothesis pruning, and merging of state estimates. Many of these methods are summarized in \cite{MHT-Overview}. 

While the MHT has seen various forms of success, it faces several key challenges. Namely, the curse of dimensionality and complexity. The number of possible hypotheses grows exponentially with the number of potential tracks and the number of scans. Consequently, it is considered intractable for large scenarios. Moreover, the MHT might require extensive tuning and thus may be difficult to implement in practice, in addition to being computationally expensive. For these reasons, it is generally considered to be one of the most complex MTT algorithms. 

A Probability Data Association (PDA) takes a Bayesian approach to solving the data association problem by finding detection-to-target assignment probabilities via a posterior PDF, which again requires heavy assumptions on object dynamics and the detection process. In similar fashion, a Joint PDA (JPDA) assigns probabilities that are computed \textit{jointly} across all targets. The JPDAF is an algorithm which implements the JPDA along with filters and estimation methods as discussed previously in \cite{Bar-Shalom_MTT}.

A limited number of optimization based algorithms have been applied to solve the MTT problem, most of which attempt to solve by mapping the measurement set onto a trellis and seek the optimal measurement association sequence. Some examples include the Multi-Target Viterbi \cite{Viterbi-1} and an extension in \cite{Viterbi-2} which formulates \cite{Viterbi-1} as a network flow, reducing the solve time from exponential to polynomial. Still others, in particular \cite{Viterbi-3}, have suggested adaptations of this approach that output a single best set of K tracks, or a list of L best sets of k tracks, similarly to the MHT.  

Compared to the number of statistically based algorithms in the MTT literature, optimization based algorithms are relatively lacking. In fact, most occurrences of optimization in the MTT literature propose the use of optimization to leverage statistical algorithms, in particular, the MHT. For example, integer optimization has been used to improve MHT hypothesis selection by solving an assignment problem which chooses the best hypothesis, but only after costs have been assigned (statistically based) and hypotheses have been pruned \cite{MHT-IP}. Somewhat similarly, linear optimization has also been used to assist in the hypothesis selection process for the MHT \cite{MHT-LP}. Still, other attempts aim to improve the MHT hypothesis selection process via Lagrangian relaxation \cite{Lagrangian}. 

More recently, Andriyenko and Schindler have proposed formulating the MTT problem as a minimization of a continuous energy \cite{Continuous_energy}, and then again as a minimization of discrete-continuous energy \cite{Discrete-Continuous_energy}. These algorithms aim to more accurately represent the nature of the problem, but sacrifice interpretability for complexity in the process. Rather than formulating the problem to lend it easily to traditional global optimization methods, the authors intend to leverage the use of optimization techniques to find strong local minima of their proposed energy objective, and they achieve strong results in doing so. However, this approach calls for the use of several parameters that must be tuned and few recommendations are provided for how to go about such a tuning process. Additionally, these methods require initialization heuristics to begin the solving process, which is in itself complicated to implement and is not directly connected to the optimization problem solved. 

In this paper, we propose the use of mixed integer optimization (MIO) to formulate and solve the multi-target tracking problem. Although MIOs are generally thought to be intractable (NP-Hard), in many practical cases near optimal solutions and even optimal solutions to these problems can be obtained very efficiently \cite{Computation}. This can be attributed to the fact that MIO solvers have seen significant performance improvements in recent years due to advancements in both methodology and hardware. The development of new heuristic methods, discoveries in cutting plane theory, and improved linear optimization methods have all contributed to improvements in performance \cite{Gurobi-MIP}. Modern solvers such as Gurobi and CPLEX have been shown to perform extremely well on benchmark tests. In the past six years alone, Gurobi has seen performance improvements by a factor of 48.7 \cite{Gurobi-Benchmark}. CPLEX saw improvements by a factor of 29,000 from 1991 to 2007 \cite{CPLEX-Benchmark}. From 1994 to 2014, the growth of supercomputing power as recorded by the TOP500 list has improved by a factor of 567839 \cite{Supercomputer}. Thus, the total combined effective improvement of software and hardware advancements is on the scale of 800 billion times in the past 25 years. 

The literature is also lacking in performance metrics for the evaluation of MTT algorithms. There is no standard method of measuring scenario complexity or algorithm performance as a function of this complexity. In many cases, only the sensor's detection noise is taken into account and other factors such as target density are negated. Recent work \cite{MTT-Performance} proposes a mathematically rigorous performance metric for measuring the distance between ground truth and estimated track, but there is not much attention given to the complexity of generated scenarios. In this paper, we also suggest measures of complexity and performance which are related to the ones suggested in \cite{MTT-Performance}, but we show the value in relating a complexity measure to performance measures, namely that it allows us to evaluate the data association and trajectory estimation problems separately. We evaluate the methods suggested in this paper using these complexity and performance measures on two simulated experiments.

The main contributions of this paper are as follows: 
\begin{enumerate}[(i)]
\item We introduce a simple interpretable MIO model which solves the data association and trajectory estimation problems simultaneously for a sensor with no detection ambiguity. The model does not require assumptions on data generation or any tuning of parameters. This MIO is practically tractable, in the sense that it obtains high quality solutions in reasonable amount of time for the considered applications.
\item We propose a simple heuristic, motivated by the optimization problem, which provides feasible solutions to this problem and show how it can be used as warm start to the MIO in order to improve the quality of the solutions obtained as well as the running time. This heuristic is highly scalable and parallelizable, solving in milliseconds.
\item We extend this basic MIO model and corresponding heuristic initialization algorithm for the case of detection ambiguity, i.e., the case where there are both missed detections and false alarms, keeping interpretability while only adding two tunable parameters, as well as provide general guidelines as how to tune these parameters. 
\item  We present a new measure of complexity for the data association problem and show how it allows scenario generation and performance to be measured separately in each of their own natural demands. We also discuss a simplified measure of performance for the trajectory estimation problem. 
\end{enumerate}

The paper structure is as follows. We begin with a description of the MTT problem as we wish to model it in \mysection~\ref{\myabrv Problem Description}. In \mysection~\ref{\myabrv Basic MIO Model} we introduce a simple MIO formulation for a sensor with no detection ambiguity and extend it to a generalized formulation. Then we present a randomized local search heuristic in \mysection~\ref{\myabrv Heuristic}, which we use as a warm start for the MIO. In \mysection~\ref{\myabrv Robust MIO Model} we discuss extensions to both the MIO model and the heuristic for the case of detection ambiguity. In order to quantify the performance of our suggested methods, we develop metrics for measuring scenario complexity and algorithm performance in \mysection~\ref{\myabrv Scenario-Performance}.  Experimental methods and computational results are presented in \mysection~\ref{\myabrv Results}, including results for scenarios both with and without detection ambiguity. Finally, we summarize our contributions and identify future work in \mysection~\ref{\myabrv Conclusion}.

{\bf General Notations:}
Unless specified otherwise, $\|\cdot\|$ is used to indicate the $\ell_1$ norm, and $|\cdot|$ refers to element wise absolute value.
In this paper, we restrict our exploration of the MTT problem to the automatic tracking of multiple, independent point targets using a single sensor. A \textit{target} is the object of interest. A point target's only identifiable attributes are features of its state space, which we restrict to position and velocity. The state space fully defines the field of \textit{trajectories}, or paths, along which targets travel. A \textit{detection} is collected from each target at sequential scans and is subject to noise. We consider two general scenarios: with and without detection ambiguity. 

When there is no detection ambiguity the sensor produces exactly one detection for each target in each scan, without any other source of detections. Therefore, the number of detections in each scan is exactly equal to the number of existing targets. Under these conditions, the data association problem reduces to a one-to-one assignment problem. Our basic optimization model, presented in \mysection~\ref{\myabrv Basic MIO Model}, addresses this variant of the MTT problem.

The presence of detection ambiguity results in a more complex case where the sensor both generates false alarms and misses detections. A \textit{false alarm} occurs when a detection is collected but in fact no target exists. This could be the result of measurement error or difficulties in the sensor's signal processing. A \textit{missed detection} occurs when a data point is not collected in a given scan where a target actually exists. Due to such ambiguity the number of detections in each scan could be higher or lower than the actual number of existing targets. Thus, the number of targets can not be immediately deduced from the number of detections. Under these conditions each detection can be assigned to either a target, in the same manner as before, or classified as a false alarm. Furthermore, we wish to identify the location (scan and target ID) of a missed detection. In \mysection~\ref{\myabrv Robust MIO Model} we present extensions of our basic optimization model to a robust formulation that deals with this detection ambiguity, which we will refer to as the robust MIO model.

Throughout the paper we make the following assumptions:
\begin{assumption}\label{ass:general_assumption}
\leavevmode
\begin{enumerate}[(i)]
\item All targets have constant velocity. \textit{i.e.}, targets do not maneuver and no outside forces act on them.
\item Each target's dynamics are independent of any other target's dynamics.
\item The number of targets remains constant throughout the window of observation, \textit{i.e.}, there is no birth/death of targets.
\item The detection errors are independent of one another.
\end{enumerate}
\end{assumption}

{\bf Notation:}
We observe $P$ targets over a fixed time window in which $T$ scans are collected. Without loss of generality, and for ease of notation,  we assume the scans arrive at a fixed rate of 1Hz, such that the set of scans can be time stamped by $\{1, 2,...,T\}. $ The $i^{th}$ detection of the $t^{th}$ scan is indicated by $x_{it}$, such that a scan of data at time \textit{t} is the unordered set of detections $\mathcal{X}_{t} = \{x_{1t}, x_{2,t},...,x_{P,t}\}$. The data for the problem is the ordered set of scans $\boldsymbol{\mathcal{X}}=(\mathcal{X}_{1},\mathcal{X}_{2},...,\mathcal{X}_{T})$. The state space of target trajectories is paramaterized by a true initial position $\alpha^{\text{true}}_{j}$ and a true constant velocity $\beta^{\text{true}}_{j}$. 
\appendix
Here we provide recommendations for the tuning of penalty parameters $\theta$ and $\phi$. We begin with an explanation grounded in logic. It can be shown that as the false alarm rate $\lambda$ increases, the number of expected false alarms also increases. Therefore, it stands to reason that as a general rule of thumb the false alarm penalty $\theta$ should decrease as $\lambda$ increases. Similarly, the number of expected missed detections increases as the missed detection probability $\gamma$ increases, and so too the missed detection penalty should decrease. Then it follows logically that the missed detection penalty $\phi$ should increase as $\gamma$ decreases. Furthermore, it is convenient to reason that the value of both of these penalties should somehow be tied to the value of $\sigma$. This is due to the fact that as the noise increases, expect a higher standard deviation for the detections. 
Thus, when assigning false alarms and missed detections, then error will naturally be higher in the objective function. To combat this effect, we should consider increasing both penalties as $\sigma$ increases. Through examination we found these logical concepts to generally hold true across a variety of scenario sizes and difficulties. Using this insight as well as the results of a mini experiment, we tuned our penalties for the robust experiment. The false alarm penalties used are shown in Figure~\ref{tab:Theta_Penalties} and the missed detection penalties are shown in Figure~\ref{tab:Phi_Penalties}.
\begin{table}[ht]
\centering
\begin{tabular}{c|m{1cm}m{1cm}m{1cm}m{1cm}}
  \hline
   & \multicolumn{4}{c}{$\sigma$} \\
   \cline{2-5}
   $\lambda$ & 0.1 & 0.5 & 1.0 & 2.0\\
  \hline
  \hline
   0.1 & 1.7 & 2.6 & 3.1 & 3.5 \\
   0.5 & 1.1 & 1.9 & 2.3 & 2.5 \\ 
   1.0 & 0.9 & 1.2 & 1.6 & 1.8 \\ 
   2.0 & 0.5 & 0.9 & 0.9 & 1.0 \\ 
   \hline
\end{tabular}
\caption{False alarm penalties ($\theta$) as a function of $\lambda$ and $\sigma$.}
\label{tab:Theta_Penalties}
\end{table}
\begin{table}[ht]
\centering
\begin{tabular}{cc|cccc}
  \hline
  & & \multicolumn{4}{c}{$\sigma$} \\
  \cline{3-6}
 $\lambda$ & $\gamma$ & 0.1 & 0.5 & 1 & 2 \\ 
  \hline
  \hline
   0.10 & 0.05 & 0.20 & 0.50 & 0.80 & 0.70 \\ 
   0.10 & 0.10 & 0.10 & 0.30 & 0.50 & 0.50 \\ 
   0.10 & 0.15 & 0.10 & 0.20 & 0.40 & 0.40 \\ 
   0.10 & 0.20 & 0.10 & 0.10 & 0.30 & 0.40 \\ 
   0.50 & 0.05 & 0.20 & 0.50 & 0.80 & 0.80 \\ 
   0.50 & 0.10 & 0.20 & 0.30 & 0.50 & 0.60 \\ 
   0.50 & 0.15 & 0.20 & 0.25 & 0.40 & 0.40 \\ 
   0.50 & 0.20 & 0.10 & 0.20 & 0.30 & 0.40 \\ 
   1.00 & 0.05 & 0.30 & 0.70 & 0.80 & 0.80 \\ 
   1.00 & 0.10 & 0.20 & 0.40 & 0.50 & 0.60 \\ 
   1.00 & 0.15 & 0.20 & 0.25 & 0.40 & 0.40 \\ 
   1.00 & 0.20 & 0.10 & 0.20 & 0.30 & 0.40 \\ 
   2.00 & 0.05 & 0.30 & 0.70 & 0.90 & 1.00 \\ 
   2.00 & 0.10 & 0.20 & 0.50 & 0.60 & 0.60 \\ 
   2.00 & 0.15 & 0.20 & 0.25 & 0.40 & 0.50 \\ 
   2.00 & 0.20 & 0.10 & 0.20 & 0.30 & 0.40 \\ 
   \hline
\end{tabular}
\caption{Missed detection penalties ($\phi$) as a function of $\lambda$, $\gamma$, and $\sigma$.}
\label{tab:Phi_Penalties}
\end{table}
For completeness here we present an alternative approach to solving the MTT problem with detection ambiguity. This MIO model is an extension to \eqref{eq:simple_robust} that directly determines the number of targets via optimization by incorporating additional variables and constraints into framework of the formulation. 

\subsection{Decision Variables}
Because we do not assume a fixed number of targets, this decision must now be made by the model. Toward this goal, we introduce a new binary decision variable $w_{j}$ to indicate whether or not trajectory \textit{j} corresponds to an existing target. Any detections assigned to a non-existing target would be a false alarm. 
\[w_{j} = 
\begin{cases}
1, & \text{if trajectory \textit{j} exists,}\\
0, & \text{otherwise.}
\end{cases}\]

\subsection{Constraints}
Most constraints remain similar to their original counterparts in \eqref{eq:simple_robust}, except for slight adjustments needed to account for the possibility that some trajectories may not exist. Explicitly, the number of possible trajectories is now $N_{1}$ so all instances of summing over \textit{j} must be changed accordingly. For example, we adjust \eqref{eqn: FA Simple} and \eqref{eqn: MD Total} as follows: 
\begin{align*}
\sum_{j=1}^{N_{1}} y_{itj} + F_{it} = 1 \qquad \forall i,t,\\
\sum_{j=1}^{N_{1}} \sum_{t=1}^{T} M_{jt} = TM.
\end{align*}

By the same accord \eqref{eq: } no longer equates to 1 because some trajectories may not exist. Therefore we say all \textit{existing} trajectories must either be assigned a detection or a missed detection,
\begin{align}\label{eqn: Existing Targets}
\sum_{i=1}^{n_{t}} y_{itj} + M_{jt} = w_{j} \qquad \forall j,t.
\end{align}

Moreover, we restrict $\alpha_{j}$ and $\beta_{j}$ to be zero if trajectory \textit{j} does not exist. This ensures only existing trajectories are penalized in the objective function. 
\begin{align*}
|\alpha_{j}|+|\beta_{j}| \leq M_{0}w_{j}\qquad \forall j.
\end{align*}

We can actually reduce the symmetry that is inherent to this formulation. Since $N_{0} \leq P \leq N_{1}$, we can set $w_j=1$ for all $j=1,\ldots,N_0$, which leaves us with only $N_1-N_0$ additional binary variables. We simply need the additional constraint
\begin{align*}
w_{N_0+1}\geq ...\geq w_{N_1},
\end{align*}
which reduces the symmetry of the formulation, and thus making it efficiently solvable. 

\subsection{MIO with Number of Targets as a Decision Variable}
Incorporating these additional variables and constraints, we arrive at the following complete alternative formulation.
\begin{align*}
\underset{\psi_{jt}}{\text{minimize: }} & \sum_{j=1}^{N_{1}} \sum_{t=1}^{T} \psi_{jt} + \theta \cdot TF + \phi \cdot TM\\
\text{subject to: }	& \sum_{j=1}^{N_{1}} y_{itj} + F_{it} = 1 \qquad \forall i,t \nonumber\\
				& \sum_{i=1}^{n_{t}} y_{itj} + M_{jt} = 1 \qquad \forall j=1,...,N_{0},t \nonumber \\
				& \sum_{i=1}^{n_{t}} y_{itj} + M_{jt} = w_{j} \qquad \forall j=N_{0},...,N_{1},t \nonumber \\
				& \sum_{i=1}^{n_{t}} \sum_{t=1}^{T} F_{it} = TF \nonumber \\
				& \sum_{j=1}^{N_{1}} \sum_{t=1}^{T} M_{jt} = TM \nonumber \\
				& w_{N_0+1}\geq ...\geq w_{N_1} \nonumber \\
				& |\alpha_{j}|+|\beta_{j}| \leq M_{0}w_{j}\qquad \forall j \nonumber \\
				& x_{it}y_{itj} + M_{t}(1-y_{itj}) \geq z_{jt} \qquad \forall i,t,j \nonumber \\
				& x_{it}y_{itj} - M_{t}(1-y_{itj}) \leq z_{jt} \qquad \forall i,t,j \nonumber \\
				& z_{jt} - \alpha_{j} - \beta_{j}t \leq \psi_{jt} \qquad \forall j,t \nonumber \\
				& -(z_{jt} - \alpha_{j} - \beta_{j}t) \leq \psi_{jt} \qquad \forall j,t \nonumber \\
			 	& y_{itj} \in \{0,1\} \quad \forall i,t,j \nonumber \\
				& \alpha_{j} \in \mathbb{R}^n,\quad \beta_{j} \in \mathbb{R}^n,\quad w_{j} \in \mathbb{R}^n \quad \forall j \nonumber \\
				& z_{jt} \in \mathbb{R}^n, \quad \forall j,t. \nonumber
\end{align*}

\subsection{Extension of Robust Heuristic}
Insert brief discussion about how this heuristic would be identical to the one proposed previously, except a number of estimated targets would be randomly selected during the initialization process and then the heuristic would progress identically to the robust heuristic already proposed.
%% This defines the bibliography file (main.bib) and the bibliography style.
%% If you want to create a bibliography file by hand, change the contents of
%% this file to a `thebibliography' environment.  For more information 
%% see section 4.3 of the LaTeX manual.
\begin{singlespace}
\bibliography{main}
\bibliographystyle{plain}
\end{singlespace}

\end{document}

